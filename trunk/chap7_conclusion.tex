
%==========================================
\section{Summary}\label{ch:conclusion_sec:1}
%==========================================
This thesis has described our research of developing intuitive
interfaces and effective algorithms for sketch-based 3D modeling and
reconstruction. Our aim is to develop modeling techniques and tools
allowing the user to effectively create, edit, and manipulate 3D
models using sketch-based interface. To this end, various techniques
for surface reconstruction and editing have been investigated. The
analysis on user behavior, psychology and intrinsic properties of 3D
shapes have been employed to guide the development of techniques for
better and high-level manipulation. We have established a framework
for sketch-based modeling and developed supporting algorithms for
creation and editing operations in the sketching interface. We also
present a solution to the problem of surface reconstruction from
cross sections. The goal of our research has been fulfilled. In
particular, our work of the research includes:

We present a reference plane assisted sketching  interface for
freeform shape design. We propose some rules for regularizing and
interpreting the user input when sketching the shape of the 3D model
to be created. These rules consider both the semantic meaning of the
sketched strokes and human psychology and partially eliminate the
ambiguity of understanding the user input sketches. We propose to
use reference planes to assist the creation and editing functions,
such that the location and orientation of strokes in 3D space become
meaningful and intuitive. These reference planes are created
automatically during the process of sketching and help make the
sketch-based modeling process intuitive and easy to use.
%Finally we propose a local refinement algorithm to improve the mesh quality after large deformation. This system also serves as a platform to implement various digital geometry processing algorithms for our research in the next steps.

We introduce a progressive approach for modeling, which allows the
user to create a 3D model by iteratively sketching cross section
curves, taking both the up-to-date reconstructed model and the
existing curves as references. In this way, perception ambiguities
on the sketched 3D shape can be eliminated to a large extent and the
sketching function in sketch-based modeling systems is enhanced to
allow the creation of complex 3D models with arbitrary topologies.
To effectively support this operation, we develop a new surface
reconstruction algorithm which enables gradual shape updating and
produces models with a singly connected component when the user
iteratively adds new sketches.

To better support the sculpting function  in our sketching
interface, we propose an edge-based flexible mesh deformation
algorithm. An edge-based graph for a triangular mesh is proposed to
evaluate more points on the mesh surface without explicit re-meshing
or re-sampling and deliver more accurate computation result. Based
on the edge-based graph, a mixed deformation model, which considers
both the changes of the first order and the second order
differential quantities, has been proposed. Various deformation
effects between local shape preservation and global smoothness can
thus be produced by tuning a simple scalar parameter. The scalar
parameter can also be applied to local regions of the mesh to
simulate the deformation behavior of real-world objects with
non-uniform materials. Experimental results have demonstrated the
effectiveness and flexibility of our approach.

We also extend our techniques to provide a solution to the
challenging problem of surface reconstruction from cross sections
with arbitrary orientations. The solution consists of two modules.
The first one is generalizing the method in our progressive modeling
to generate sub-surfaces for irregular zones and to extend some
sub-surfaces to connect possibly isolated surface components. The
second one is further editing of local topology of the initially
generated model. By utilizing the intermediate results obtained
during the surface reconstruction process instead of from scratch,
we provide the user a simple sketching tool to quickly edit the
local topology of the generated model if the initial shape is not
satisfactory. As a result, the final output model of our surface
reconstruction is guaranteed to be a manifold surface having only
one connected component, which is usually difficult to achieve in
previous methods. Various examples have demonstrated the usefulness
and efficiency of our approach.


%==========================================
\section{Future directions}
\label{chap:conclusion_sec:2}
%==========================================
%more references
In our sketching interface, we have provided users planes as
references for mapping the 2D sketches into 3D, such that the user
input will become intuitive and meaningful. There exist other types
of geometric objects such as sphere and cylinder, which can be taken
as references, to make the user interface intuitive and flexible.
Nevertheless, with the introduction of more reference types, the
intelligent selection of references may become a challenge. This
thus requires careful evaluation on the trade-off since each type of
references has its own pros and cons. Moreover, the combination of
different types of references may be considered.

%shape description
Currently the input in the sketching function of our system  is
allowed to be the cross section curve, which can be quickly sketched
and provides an intuitive and accurate description of a 3D shape.
However when an artist/designer sketches a 3D model, other strokes
which depict the features, such as the hatching, shading, and
suggestive contours~\cite{DFRS03} are also used. These shape
descriptors may also be incorporated into our sketching function.
This involves not only how to interpret the user sketches, but also
how the progressive surface reconstruction algorithm can be improved
to have more types of input curves besides the cross sections.

%Feature-sensitive mesh deformation
For the mesh deformation algorithm, most existing works  tried to
maintain various local and low-level geometric properties of the
mesh, while some recent works~\cite{GSMC09,ZFCAT11} have shifted the
focus to high-level shape editing which preserves the structural and
semantic shape features of man-made objects. However for free-form
objects, this remains a big challenge. There have been various
feature descriptions proposed, such as ridges and
valleys~\cite{OBS04}, and suggestive contours~\cite{DFRS03}, though
they usually do not have the regular relationships as the features
in the man-made objects. After these features are extracted and
analyzed, the corresponding deformation algorithms on preserving
these features should be explored.

%skeleton guided
For the problem of surface reconstruction from cross  section
curves, currently the topology of the output model is initially
decided by the relative positions of the cross sections. Though we
provide a sketching tool for the further modification of the
topology, it still seems troublesome when the target one is far more
different from the initial shape. In that case, some other
descriptors which suggest the target topology can also be included
as the input besides the cross sections. For example, it will be
interesting to allow the user to sketch the skeleton of the target
shape which could suggest the desired topology. Meanwhile, template
models which have similar topologies can also be utilized to guide
the reconstruction. Correspondingly, our surface reconstruction
algorithms should be modified to incorporate more input information.

%subdivision
Moreover, the surface reconstruction algorithm can also  be extended
to produce subdivision surfaces which have higher quality and
scalability than mesh surfaces. The output in that case will be a
coarse control mesh, whose limit surface under subdivision rules
interpolates or approximates the input cross sections. The
challenges will include how to build the initial control mesh and
how to reduce the number of extraordinary points which are not
preferred in subdivision surfaces. This technique may also be used
to help create high quality 3D models, which is a challenging
problem in sketch-based modeling.


%edge-based graph
%two hands
