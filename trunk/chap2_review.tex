This chapter reviews the state-of-the-art in the related research
areas.  We begin from the sketch-based modeling, especially the two
important modules for producing 3D models: sketching and sculpting.
Then we survey the work on mesh deformation and surface
reconstruction from cross sections, which are two fundamental
geometric algorithms in sketch-based modeling and reconstruction.



%-----------------------------------------------------------------------------------------------------------------
\section{Sketch-based modeling system}\label{ch2:sec:sbim}

In recent years, a lot of papers on sketch-based  modeling have been
published~\cite{OSSJ09}, proposing different kinds of user
interfaces and algorithms. Some of these works make use of existing
template models or images to help interpret the user input and
reconstruct the target models, while others build 3D models purely
from the 2D user sketches without using any prior information.

The template-based systems are characterized by  the fact that they
have some \lq\lq{memory}\rq\rq of 3D shapes built in, which guide
their interpretation of the input sketches. This requires the
analysis of the geometric or semantic features of the template
shapes, such as the silhouette of the 3D shape under some views, the
curvature distributions, etc. These methods are usually dedicated to
some specific modeling applications, such as the modeling of
clothes~\cite{TWBCH07}, trees~\cite{TST06,OOI06,CNXDK08},
hair~\cite{WBC07}, clouds~\cite{WBC08}, architectural
shapes~\cite{SketchUp} and so on.

Contrarily, the other systems take only the user  sketches as input
and build 3D models from scratch. In such systems, sketching and
sculpting, which are the main techniques used by humans to
communicate about shapes~\cite{CIW08}, are adopted for the creation
and editing of the 3D models. In the sketching method, the user
sketches the 2D curves from scratch and the 3D models with the
depicted shape will be constructed automatically. While in
sculpting, the user starts the creation from a very rough 3D shape
and uses various sketching tools to carve, deform, extend or smooth
it to a delicate one. Here we first review the systems using the
sketching technique, and then examine various sculpting tools in
detail.


%-----------------------------------------------------------------------------------------------------------------
\subsection{Sketching methods}
\label{ch2:sec:sbim:sketch}


The sketching technique usually lets the user  sketch the
silhouettes and feature lines to depict the 3D shape and then
constructs the model automatically.

\textbf{Single view sketching}: One popular  sketching method is to
first allow the user to complete complex sketches on the 2D plane
under a fixed viewpoint, then interpret and map the 2D curves or
strokes into 3D space according to some rules, and finally
reconstruct the 3D model corresponding to the input sketches. This
is quite similar to people's drawing behavior when sketching an
object on a paper with a pen. The way of the interpretations usually
conforms to the 10 visual rules proposed by Hoffman~\cite{HD00} for
the understanding of the 2D drawings which depict a 3D object in
visual intelligence.

%single view, balloon-like simple shape
The pioneer work Teddy~\cite{IMT99} allows the  user to sketch a
simple closed contour to depict a model with balloon-like shape
which takes the curve as its silhouette. This is consistent with the
visual rule in~\cite{HD00} that people tend to interpret a curve as
the rim of a 3D surface which is as smooth as possible. Then the
closed area bounded by the curve is triangulated and the triangle
vertices are elevated with the extracted chordal axes to construct a
balloon shaped 3D mesh. Similar methods can also be found
in~\cite{IH01,IH03}. FiberMesh~\cite{NISA07} then improved this
method to get a mesh surface with higher quality and smoother shape,
by using the method similar to~\cite{NPAI09} to get the initial
triangles with more regular shapes and minimizing two quadratic
energy functions to iteratively optimize the mesh surface.

Olga Karpenko et al.~\cite{KHR02} proposed a  similar system to
Teddy, but they used the RBF(Radial Basis Function) implicit surface
instead of mesh as the surface representation. By taking the points
on the user sketched curve as the \lq\lq{zero points}\rq\rq which
correspond to the zero value of the basis function, and generating
some extra points which indicate the outside and inside information
of the target surface, the parameters of the implicit surface can be
obtained. Similar methods can be found
in~\cite{AJ03,AGB04,TZF04,SWSJ05,SS08,WBC08,BMSdV10}. The implicit
surface is naturally smooth and no extra surface optimization is
needed. Meanwhile, some further editing such as the merging of
multiple surface components can be easily implemented. However,
spurious local shapes may be produced due to the intrinsic property
of the implicit surface, and the troublesome conversion to
triangular mesh is also inevitable.

To build surfaces with higher quality,  Nasri et al.~\cite{NKS09}
proposed a method of constructing a subdivision surface from a
closed 2D contour. By making use of the polygonal
complex approach~\cite{NA00}, it is able to build the control mesh of a
Catmull-Clark subdivision surface with simple topology and the limit
of the surface will interpolate the input contour. Cherlin et
al.~\cite{CSSJ05} presented a system which could generate some
special parametric surfaces from the sketched contour, such as
rotational blending surfaces and cross section blending surfaces, by
extracting the skeleton of the input curve and taking it as the axis
of the target revolution surface. The time for constructing objects
with relatively complex shapes is usually quite long, making the
system unsuitable for interactive modeling applications.

Meanwhile, all these methods are only  able to build a smooth object
with a simple shape, while the creation of the surface features in
these systems mainly relies on the subsequent editing operations.

%single view, complex shape
To produce a complex 3D model with features  on its surface through
sketching, more curves need to be sketched and more rules should be
proposed to interpret these input curves.

The information of intersection of strokes  usually helps to
interpret the sketches and reconstruct the models. For example,
CrossSketch~\cite{ASN07} proposed a method which is able to
construct a surface patch with details on it from a few strokes.
This approach detects the intersections of the input strokes and
makes use of the Cubic Corners method~\cite{DP1968} to infer the 3D
vertex positions. The 3D surface patch and details which are
consistent with the 2D input strokes under the viewpoint of
sketching can then be reconstructed sequentially. However, it is
only able to construct an open surface rather than a closed 3D mesh,
and the Cubic Corners method is not efficient for inaccurate
drawings. SmoothSketch~\cite{KH06} used the algorithm proposed
in~\cite{WH96} to detect the T-junction points from user sketched
contours and infer the hidden contours on the target surface. A
smooth surface with balloon-like shape and T-junction points under
the current viewpoint can then be reconstructed. A similar method
can be found in~\cite{CS07}.

Different from the free-form surfaces, the  CAD models are
characterized by the hard edges, rigid corners and planar
faces~\cite{OSSJ09}. The reconstruction of such models relies on the
interpretations of the sketched straight lines. These
interpretations are also consistent with those proposed
in~\cite{HD00} for the understanding of the straight lines of human
being's.

The most important step for the reconstruction  of CAD models is the
identification of the vertices, edges and corners from the 2D user
sketches. Early works usually classify line segments from
images~\cite{MJ86} or let the user to specify them
manually~\cite{PD92}, which seem tedious for interactive
applications. Later approaches improved it by either implementing
the reconstruction incrementally with each input
sketch~\cite{CNJC03}, or using some rules or templates to create the
model after all the input sketches complete~\cite{FBSS04,VMS05}.
However, these systems are limited to construct objects consisting
of only straight lines, and no curves are allowed.

Varley et al.~\cite{PYJH04} gave an initial  attempt on including
curves into the sketch-based modeling of CAD models. First of all
straight lines are required to create a basic frame of the model,
then the user is allowed to give some further sketches to modify the
shape of the model and let it contain curves. Masry et
al.~\cite{ML07} then made this process fully automatic. First they
selected a vertex with three attached lines as the origin and used
the optimization method proposed in~\cite{KML04} to reconstruct the
depth of all the endpoint vertices of all strokes. Similar algorithm
was also used in~\cite{CCCP04}. Then the gradient steepest descent
algorithm was used to reconstruct the depth of the curves. This
method was further improved in~\cite{LF12}, which combines the
optimization and the Cubic Corners methods to balance between the
correct convergence and the efficiency for inaccurate input
sketches.

Although the single view sketching methods  provide a natural and
simple way for the user to create a 3D model by making use of
various rules to map the 2D user sketches into 3D, ambiguities on
the interpretation still exist when the sketches become complex.
Meanwhile, these methods also propose various requirements on the
user sketches to guarantee the validity of the input, which limit
the user's creativity and also the shape of the reconstructed object
to some extent.


\textbf{Multi-view sketching}: To avoid the  ambiguity problem in
the single view sketching systems and allow the user to give more
descriptions on the target shape, some methods have been proposed to
let the user sketch the outlines of the model from multiple
viewpoints.

Kang et al.~\cite{KMR04} presented a system  which allowed the user
to sketch 2D curves under two different viewpoints and used the
Epipolar geometry to combine them into a 3D curve, such that more
precise description of a target 3D curve can be obtained. Though
being a little troublesome, this method provides a feasible way to
define the shape of the target model through sketching multiple
curves.

Das et al.~\cite{KPG05} allowed the user to  interweave the
sketching and viewpoint adjustment to build surface patches from
sketched boundary curves. Similar method was adopted
in~\cite{KDS06,KS07,BHSF09}. Under each viewpoint, the input 2D
curve will be mapped to a 3D curve which has the minimum normal
curvature among all the potential 3D curves whose projections
overlap with the 2D curve. Thus, the 3D curve will be as smooth as
possible. After that, the bounding curves of each surface patch are
identified manually, and the final surface can be obtained by
stitching these patches. Abbasinejad et al.~\cite{AJA12} further
made the identification of the bounding curves of each patch fully
automatic. The patch-based method was extended in~\cite{BBS08}
and~\cite{BBS09} which employed a multi-touch display and defined
multiple gestures to make the user interface more intuitive and
flexible.~\cite{OK11} and~\cite{RSWCT07} further made use of these
approaches to build subdivision and developable surfaces
respectively. These patch-based methods work well on producing
non-manifold surfaces, while for manifold surfaces, the
reconstruction will be more involved.

Sowel et al.~\cite{SLJGAGL09} proposed a  system allowing the user
to manipulate a virtual view plane to intersect with the given 3D
medical data and sketch cross sections along the contours on each
plane. Then the algorithm proposed in~\cite{LBDLJ08} was used to
reconstruct a closed mesh surface which interpolates the
non-parallel input curves. This method relies on the predefined
volume data, while for sketch-based free-form modeling applications,
the input strokes may not be regular, and the analysis and process
of them will thus become complicated.

More recently, Rivers et al.~\cite{RDI10}  presented a system which
has a similar interface to the traditional modeling software. The
user sketches the silhouettes of a 3D object from three orthogonal
views and the final model is computed by using a method similar to
the visual hull approach~\cite{LA94}. This method requires much
imagination for novice users when a model with multiple components
and different depths are desired.

Sketching a 3D model from multiple viewpoints  gives more
descriptions of the target shape and avoids the ambiguity problem in
single view sketching to a large extent. Nevertheless, it is not
easy to maintain the coherence and validity of the sketches from
different viewpoints, especially for shapes with complex topologies.

Despite all the above mentioned attempts on  reconstructing 3D
objects from 2D sketches, the modeling result can only be obtained
after all the curves have been sketched, thus little hints of the 3D
shape are provided to the user during the sketching process.
Meanwhile, sometimes it is still difficult to create a 3D object
with many details in the sketching methods. The underlying reason is
that sketching does not directly communicate the shape people have
in mind, and such drawings take a great deal of skill which
restricts the medium to trained and experienced
designers~\cite{CIW08}. Therefore, some subsequent editing
operations, which are similar to the sculpting behavior in real
life, are also important for further modification of the initially
reconstructed model in sketch-based modeling systems. Next we will
introduce the sculpting operations.


%-----------------------------------------------------------------------------------------------------------------
\subsection{Sculpting methods}
\label{ch2:sec:sbim:sculpt}

The sculpting method provides various tools to  allow the user to
modify the shape of the model and add features on the surface
through iterative 2D sketches and gestures. Since the editing
operations are implemented directly on the model surface, perception
ambiguities on the input sketches and output shapes can be avoided
to a large extent. Therefore, the sculpting methods are quite
popular and compose a large body of the works on sketch-based
modeling.

The sculpting technique can be thought of as the  melding of two
parts: the original surface and the sketched features. According to
the scale of the features, the sculpting operations can be divided
into three categories: The big scale feature creation which is used
to modify the global shape of the model; The small scale surface
augmentation which focuses on creating or erasing detailed features
on local parts of the surface; The surface deformation technique
which produces surfaces with appealing shapes without introducing
any additional features -- all elements in the final surface come
from the original surface.

%big scale: extrusion, boolean, cutting,
To add a surface part with big scale to the existing  surface,
Teddy~\cite{IMT99} provided an extrusion operation to define a new
part connecting the original model. It is a two-stroke operation: a
closed stroke sketched on the surface suggesting the part to extrude
and a stroke depicting the silhouette of the extruded surface
sketched on the view plane perpendicular to the former stroke. Then
the closed stroke will be swept along the skeleton of the silhouette
stroke to construct the extruded 3D shape. Similar functions can be
found in~\cite{NISA07,MI07,WM07}.

To give a more detailed description to the shape of  the part to
add, Karpenko et al.~\cite{KHR02} proposed to allow the user to
first sketch the silhouettes of the new part and then merge the two
parts by sketching a guidance stroke. The use of the implicit
surface as the surface representation makes the computation much
easier. This approach was also used
in~\cite{AJ03,AGB04,TZF04,SWSJ05,SSB08,WBC08}.~\cite{WM07,RDI10,LMZ11}
further extended this operation to mesh surfaces by using the
CSG-based methods.

To remove a surface part, the cutting~\cite{IMT99,NISA07,MI07,LMZ11} and tunneling~\cite{IMT99,NISA07,SWSJ05,SSB08} techniques are often used. The former allows the user to cut a surface part off by sketching the cutting boundary, which is a simplified version of the operation usually seen in interactive mesh segmentation~\cite{SA08}. The latter is used to dig a hole by sketching two closed circles on the frontal and back sides of the model.

%small scale: feature creation, smoothing
The systems~\cite{NSAC05,OSSJ05,SWSJ05,NISA07} provided  functions
for the user to create sharp or smooth features with small scales on
the current mesh. The user drawn stroke is first projected onto the
existing mesh surface, then the features are created along the
projected stroke by moving the related vertices. Furthermore, the
projected sketch stroke can be seen as new positional constraints
for some advanced surface optimization
operations~\cite{NISA07,EP09}. Yotam et al.~\cite{GZ08} enhanced the
feature creation function by using the shading information. It
allows the user to sketch shadings on a 3D shape under the current
viewpoint to create different kinds of features, just like what the
artists do when drawing 2D pictures. Instead of creating the details
from scratch, Zhang et al.~\cite{ZCFWP10} allowed the user to sketch
around the existing features on the mesh surface to enhance them, by
making use of a variation of the bilateral
filter~\cite{JDD03,FDC03}.

Due to the inaccuracy of the user sketches and iterative  operations
during the modeling process, some local parts on the model may not
be smooth. The systems~\cite{IMT99,NISA07,LMZ11} enable the user to
sketch back and forth to smooth these local parts, by using various
mesh smoothing algorithms.


%deformation
Note that all the above sculpting operations aim at changing  the
features on the original surface, regardless of the scale of the
modification. To modify the shape of the model while preserving the
original features, the deformation tool, which is a very effective
and popular function in surface sculpting, is often used.

Existing deformation methods can be divided  into two categories:
the space-based deformation~\cite{GB08} and the surface-based
deformation~\cite{BS08,XZ09}. The former methods indirectly reshape
an object by warping its surrounding space. They are computationally
efficient and applicable to a variety of object representations.
Since manipulating ambient space directly is infeasible,
deformations are controlled by tools of various dimensions, such as
points~\cite{BK05}, curves~\cite{SF98,PJF97} and
volumes~\cite{SP86,JSW05,LLC08}. However in these methods, when
large scale deformation happens, the geometric details on the
original surface could not be well preserved. Besides, it is not
suitable for applications where direct manipulations on the mesh
surfaces are needed.

The surface-based method allows the user to deform the  mesh by
manipulating a handle on it.~\cite{NISA07,SCLARS04,SA07,SWS10}
proposed to let the user drag a point on the surface along the
direction parallel to the screen and then compute its new position
according to the position of the camera. The new positions of the
surface part to deform is then computed according to the positional
change of the handle point by solving an optimization problem. This
approach can give real-time feedback of the shape change to the user
and is suitable for interactive editing, though it is not intuitive
enough for confirming the direction of deformation in 3D space and
the degree of freedom on the manipulation is limited.

To manipulate multiple handle points at the same time,  Nealen et
al~\cite{NSAC05} proposed to deform a mesh by changing the shape of
a handle curve on the surface. First, a curve is drawn on the mesh
to serve as the handle curve. Then the model is rotated manually to
a proper angle and the deformed curve is drawn on the screen. The 3D
position of the new curve could be calculated according to the
current viewpoint and the deformed mesh can be computed.
SilSketch~\cite{ZNA07} extends this idea by automatically detecting
the handle curve which is a part of the silhouette of the model and
corresponds to a user input stroke for the deformed handle curve. A
similar method can be found in~\cite{KSV09}. Since these methods
rely on the 2D screen and lack references for mapping the 2D
information into 3D, the handle curve and its deformed shape are
limited to be planar.

The details of the algorithms on the surface deformation  will be
introduced in Section~\ref{ch2:sec:deformation}.

Different from the sketching technique, the sculpting tools  allow
the user to directly interact with the model surface. However when
the initial shape is far from desired, iterative and boring
sculpting operations need to be done. Meanwhile, it will not be easy
to preserve the quality of the final shape if too many operations
are implemented.














%-----------------------------------------------------------------------------------------------------------------
\section{Surface-based mesh deformation}\label{ch2:sec:deformation}

Deformation is an important tool for the editing of a created model.
The deformation methods for parametric or subdivision surfaces have
already been developed for years, but these algorithms are difficult
to be applied directly to meshes. Due to its importance in various
kinds of applications such as geometric modeling and computer
animation, mesh deformation has become a popular research topic in
digital geometry processing.

As introduced in Section~\ref{ch2:sec:sbim:sculpt},  the
surface-based deformation method allows the user to directly
manipulate the mesh surface and is thus more suitable for
interactive applications than the space-based method which deforms a
mesh by indirectly manipulating its surrounding space.

The general goal of the surface deformation is to preserve the
local details on the original mesh while keeping the global surface
as smooth as possible. To achieve this goal, the mesh is usually
encoded into some kinds of representations which reflect the
geometric properties of the surface. The deformed mesh will then be
calculated by preserving these representations as much as possible.
Based on the way of the encoding, surface-based deformation can be
further classified into multi-resolution and single-resolution
methods.

The multi-resolution
methods~\cite{ZSS97,GSS99,KCVS98,KVS99,SL99,BK04,MS11} encode the
mesh into a base layer and several levels of refinement, according
to its surface geometric properties. The base layer is the
low-frequency component which represent the rough shape of the
surface and is typically represented in Cartesian coordinates. The
refinement are described locally to represent the geometric details
of the surface. During the deformation process, only the low-level
layers are deformed and the high levels with more details keep
fixed, such that the details of the mesh will be preserved. Using
this representation, deformation can be performed on an appropriate
user-specified level-of-detail. The multi-resolution approach is the
straightforward representation of the details of the shape in a
manner that is invariant to the global coordinate system. However,
it requires the user to manually set both the base mesh and levels
of refinement for meshes with complex details, which seems quite
tedious for interactive applications which require fast and
convenient operations.

Different  from the multi-resolution method, the single-resolution
approach encodes the geometric details of a mesh in some
measurements in a uniform manner. Since these measurements are
computed by local operators, they naturally contain the local
relationship between primitives (usually be vertices) and their
one-ring neighbors. By preserving these measurements, the mesh
details can be preserved and the global shape will be smooth. The
user can thus easily implement the deformation by directly
manipulating the mesh surfaces and no extra manual intervention is
required. These advantages make the single-resolution method a
natural choice for the interactive geometric editing applications,
such as the sketch-based modeling.

The way of  representing the mesh surface in the single-resolution
approach comes from that of representing a smooth surface. It is
known from differential geometry~\cite{MPDC76} that the first and
second fundamental forms are often used to measure geometric
properties of a smooth surface, such as lengths, areas, and
curvatures. By minimizing the changes of the two fundamental forms
and adding positional constraints for the corresponding surface
parts, the geometric properties of the original surface can be
preserved during a deformation process, leading to the minimization
of the famous thin shell energy~\cite{TPBF87}. This energy is
physically accurate to simulate the deformation of the surface of a
real-world object made of physical material. Since the solving of
this nonlinear minimization problem is computationally expensive, it
can be further simplified by replacing the changes of first and
second fundamental forms with the first and second order partial
derivatives of the displacement function w.r.t. each point on the
surface~\cite{CG91,WW92}.

For a mesh  surface which can be regarded as a piecewise linear
surface, different methods are adopted to first represent the
geometric properties corresponding to those of a smooth surface. To
perform the deformation, the new positions of some vertices are
designated and the remaining vertex positions are solved by fitting
the new geometric properties of the deformed mesh to those of the
original mesh. The vertices whose positions are to be specified
usually include the \textit{handle} vertices, which the users want
to move to designated new 3D locations and the \textit{static}
vertices, which are expected to stay fixed during the deformation
process. The other vertices whose positions need to be solved are
usually called the \textit{Region of Interest} (ROI) vertices.

We next divide  these methods into two categories: the
Laplacian-based methods and the other methods. The former formulate
the deformation problem based on the Laplacian coordinates, which
represent the second order discrete differential quantities of the
mesh, while the latter try to solve the problem by encoding the mesh
surface into other representations and preserving them.


%-----------------------------------------------------------------------------------------------------------------
\subsection{Laplacian-based methods}\label{ch2:sec:deformation:Lap}

The most popular methods  to approximate the deformation energy on a
mesh surface are the Laplacian coordinates methods, which used the
Laplacian vectors defined on the mesh surface to approximate the
second order differential quantities of the smooth surface. These
methods comprise a large body of work on mesh deformation over the
recent years.
%The minimization of the changes of the Laplacian vectors can then be used as the approximation of that of the second derivatives in the thin shell energy of a smooth surface.

Specifically, a Laplacian  vector at a vertex is defined as a vector
pointing from the weighted center of its neighboring vertices to the
vertex itself. Its magnitude and direction approximate the mean
curvature and normal direction at that vertex respectively. The
cotangent weights scheme proposed in~\cite{MDSB02} is often used for
computing the weights for each neighboring vertex. Usually, the
weights are calculated on the original mesh and treated as constants
during the deformation, and the vertex positions of the deformed
mesh can be calculated by solving a linear system. This method was
first used by Marc Alexa in~\cite{AM01}.

However, the Laplacian vector  is a 3D vector that is not invariant
under rotation and scale transformations. So if a deformation that
contains rotation or scale happens on a given mesh, preserving the
orientation and magnitude of the Laplacian vectors w.r.t. the global
coordinate system will cause the distortion of the local details.
Thus the transformations of the Laplacian vectors during the
deformation process are also considered in subsequent works. These
works can be classified into the linear and nonlinear methods.

The linear methods represent the  transformation of the Laplacian
vector as a linear one, such that the solving of the result system
can be easy and fast. Depending on the way of estimating this
transformation, the linear methods can be further classified into
explicit and implicit types. The former methods specify the
transformation explicitly from existing information, while the
latter methods evaluate the transformation implicitly from the
deformed surface.

Yu et al.~\cite{YZXSBGS04}  proposed an explicit method of
propagating the transformation of the handle vertices to the other
vertices of the mesh using geodesic interpolation. Besides the
positional constraints, the user has to specify the transformation
matrices for the handle vertices explicitly. The transformation
matrix is then decomposed into rotation and scaling matrices and
interpolated over the ROI vertices of the mesh according to their
geodesic distances to the handles.

This method is also used in~\cite{ZHSLBGS05}, which  constructs a
graph representing the volume inside the input mesh and encodes the
volumetric details by computing the Laplacian coordinates on the
graph. By preserving these Laplacian vectors, the unnatural changes
in volume can be prevented during the deformation process. However,
when more vertices and details are involved in the deformation,
constructing the volume inside the mesh and solving the resulting
system may become quite time-consuming.

Instead of using geodesic interpolation,  Zayer et al.~\cite{ZRKS05}
proposed to use the harmonic interpolation, which computes a smooth
harmonic scalar filed on the mesh to control the propagation of the
transformation. This method shows smoother distributed local
transformations and produces better result than the geodesic
propagation method.

Popa et al.~\cite{PJS05} further used a  scalar field to control the
weight in the harmonic propagation. The scalar filed which is
specified by the user or learned from examples can be used to
specify the stiffness of the material of the object, and the
deformation of objects with different materials can thus be
simulated.

All these explicit methods require manual  specification of the
transformation at the handles by the user. Moreover, since only
rotation and scaling are considered in the transformation, the
change of orientation caused by the pure translation of the handle
vertices cannot be obtained. This will lead to the distortion of the
local shape features.

Different from the explicit methods, the  implicit methods try to
estimate the transformation automatically from the unknown vertex
positions of the deformed surface.

Lipman et al.~\cite{LSCLRS04} proposed to  estimate the local
rotation of the Laplacian vector for each vertex from the original
mesh and an initially deformed result calculated using the method
in~\cite{AM01}. Then the Laplacian vectors are rotated and a new
deformed mesh is calculated using these rotated Laplacian vectors.
This process is repeated for several iterations until satisfactory
result is obtained. This method of estimating the rotation only
works well for relatively smooth surfaces with no largely protruding
features. Besides, since scale is not considered in the
transformation, this approach cannot preserve the details well under
stretching or shrinking.

Sorkine et al.~\cite{SCLARS04} proposed  to explicitly represent the
transformation of the Laplacian vectors with a linearized matrix
which is a combination of the initial and deformed vertex positions,
and the deformed mesh can be calculated by solving a linear system.
Nealen et al.~\cite{NSAC05} later used this method in a sketching
interface. By adjusting the weights on the preservation of the
Laplacian vectors and positional constraints, various deformation
results can be obtained. This method works well for deformations
with small rotations. However, since rotation in 3D space is
inherently nonlinear, the linearization of the transformation matrix
makes the method difficult to handle large rotations in a
deformation task.

To overcome this problem, Fu et al.~\cite{FAT07}  uses a two-step
strategy to estimate the local transformation. Instead of limiting
the transformation matrix to be linearized, this method allows it to
be any affine-transformation. In the first step, this method
directly solves the affine transformation matrix associated at each
vertex. In order to obtain visual-pleasant deformation result, in
the second step, a similarity transformation of each vertex is
derived from the solved affine-transformations of its neighbors and
then used to transform the Laplacian coordinates for the subsequent
mesh reconstruction. This approach enables larger rotations
than~\cite{SCLARS04}, but it requires tweaking the relative
weighting of local transformation smoothness terms; moreover the
formulation of implicit transformations may be ill-conditioned for
flat one-rings, in which case an extra perturbation is required.

The linear methods are popular due to the robustness  and ease of
implementation. However, for the sake of speed and robustness, these
methods linearize the inherently nonlinear 3D rotation problem.
Thus, the computed result will be suboptimal.

Different from the linear methods, the nonlinear  approaches
formulate the deformation problem as a nonlinear one and use various
methods to iteratively solve it.

Huang et al.~\cite{HSLZWTBGS06} considered the  skeleton, volume and
projection constraints in addition to the positional constraints for
the vertices based on the Laplacian coordinates method, and the
result system becomes a nonlinear one. Then the Gauss-Newton solver
is used to solve the energy function iteratively. To make the
convergence more steady and faster, it projects the original energy
function onto a control mesh to reduce the number of unknown
variables. However, since the mean value coordinates it used for
editing the constructed control mesh is not a local scheme, a slight
editing of the handle vertices may be propagated to the whole mesh
and introduce undesired distortions.

To avoid this limitation, Au et al.~\cite{AFTC07}  proposed to
represent the mesh surface by a set of isolines of the scalar
harmonic field propagating from the handle vertices. These isolines
function as a constraint similar to the skeleton constraint
in~\cite{HSLZWTBGS06} and the local rigidity of the vertices can
then be measured and controlled with respect to the isolines.
However, the insertion and deletion of handles become expensive
operations in this approach, since recomputation and refactorization
of the resulting linear system are needed.

Au et al.~\cite{ATLF06} extended the Laplacian  coordinates into the
dual mesh domain. Since the valence of a vertex in the dual mesh is
always three, the shape structure formed by a dual mesh vertex and
its 1-ring neighborhood is simple, making the convergence more
stable. By keeping the magnitude of each Laplacian vector fixed and
iteratively updating the normal direction, the deformed mesh can be
calculated. Nevertheless, the convergence in this method is slow and
it is even difficult to decide when the iteration should stop since
no specific energy to minimize is formulated.

These nonlinear  methods try to reach optimal solution of the
deformation energy function and avoid the inaccurate calculation in
the linear methods. However since the computation is relatively
expensive, real-time feedback cannot be given to the user when a
large number of vertices are involved in the deformation.

Meanwhile,  since the Laplacian vectors are related to the second
order differential properties of the mesh surface, methods on
preserving them only simulate part of the thin shell deformation
energy of smooth surfaces. As a result, the deformation result may
not be satisfactory.


\subsection{Other methods}\label{ch2:sec:deformation:other}

Besides the Laplacian-based methods,  there are also some methods
proposed to encode the mesh surface into other representations and
preserve them during deformation.

To avoid the rotation-variant  property of the Laplacian
coordinates, Lipman et al.~\cite{LSLC05} proposed a frame-based
representation which is invariant under rigid transformation. This
approach first defines a discrete frame at each vertex and records
the relative relationship between frames of neighboring vertices
into a matrix. By preserving this relative relationship which is
rotation-invariant during the deformation, the frames of the
deformed surface can be estimated and the new positions of the
deformed surface can be solved. This method can preserve the details
for mesh deformation under rigid motions, and it was further
improved in~\cite{LCGL07} to handle rotations larger than $2\pi$.
This method was also used in the animation editing
in~\cite{XZYTPG07}, in which a linear least-square deformation
solver is used to solve the transformation and vertex positions
until convergence. However, since the estimation of the local frames
is separated from the positional constraints, rotations caused by
the translation of the handles cannot be detected, making this
method translation-insensitive.

Alla Sheffer et al.~\cite{SK04}  introduced the pyramid coordinates
to describe the local shape properties. A projection plane for a
vertex and its neighbors is first defined, and the edges between the
vertex and the neighbors are then projected onto the plane. The
local shape is represented by the angles between the projected
edges, the angles between the normal and the edges, and the
projected edge lengths. Given several fixed vertex coordinates as
boundary conditions, the deformed mesh can be obtained by preserving
the initially calculated pyramid coordinates, and the local details
on the original mesh can be preserved. However, the computation of
this method is too expensive.

Different from the Laplacian  coordinates methods which tried to
preserve the second order differential properties of the mesh
surface, Sorkine et al.~\cite{SA07} presented a nonlinear method to
preserve the first order differential properties during deformation.
Its basic idea is to use the edge vectors at each vertex to
represent the first order differential property at that point, and
minimize their changes under rotation transformation. Given an
initial guess of the deformed vertex positions, the mesh vertex
positions and the rotation matrix can be iteratively updated until
convergency. This method can preserve the local details under rigid
transformations, while the deformed shape usually appears too rigid
and the smoothness is not well guaranteed especially when stretching
or shrinking occurs.

To better approximate the thin shell  deformation energy on
triangular meshes, Eigensatz et al.~\cite{ESP08} proposed a
nonlinear method to minimize both the changes of the first and
second order differential properties of the mesh surface.
Especially, these two properties are represented as the inner angles
of each triangle and the two principal curvatures at each vertex
respectively. The minimization of the energy function is then solved
iteratively by using the Levenberg-Marquardt algorithm~\cite{MNT04}.
This method is further extended in~\cite{EP09} to consider the
positional, metric and curvature constraints in a deformation
process. Although the Levenberg-Marquardt algorithm is faster and
obtains better convergence than the Gauss-Newton method, this
algorithm is still quite slow to converge when the number of
vertices becomes larger.

Some recent works proposed to use  the analyze-and-edit strategy to
preserve the high-level properties of a surface, such as the global
structure and specific features. For example, Zhou et
al.~\cite{ZWS11} proposed to first extract the ridge and valley
information represented as the principal curvatures~\cite{HPW05},
and then deform the mesh by using the method in~\cite{ESP08}, such
that the feature lines of the mesh surface can be well preserved.
Gal et al.~\cite{GSMC09} proposed to extract groups of lines to
represent the shape of the man-made object. By analyzing and
maintaining the individual and mutual relationship of these lines
during the deformation, the global structure of the model can be
preserved. Similarly, the work~\cite{ZFCAT11} proposed to decompose
the model into components and use the volumes surrounding each
component to represent the shape of the object. The deformation is
then carried out by maintaining the mutual relationship of these
volumes, and the shape of the model can be preserved at the
component level.












%-----------------------------------------------------------------------------------------------------------------
\section{Surface reconstruction from cross sections}\label{ch2:sec:surfreconst}

Surface reconstruction  from planar cross section curves has been
researched for years. It refers to the process of reconstructing a
smooth surface interpolating a set of input cross sections, either
parallel or with arbitrary orientations. It has wide applications in
computer graphics and computer-aided design, such as bio-medical
modeling, terrain modeling and sketch-based modeling.

%early works,one cross section per slice
Early  works on mesh surface reconstruction~\cite{KE75, FKU77,
CS78, Kd82, GD82, WA86} focus on handling the interpolation of
parallel cross sections, for example, each of which lies on one
slice of the medical data scanned using CT or other devices. In
these works, strategies are proposed to find polygonal tiling
between each pair of cross sections on neighboring slices. While
for slices containing multiple cross sections, these methods do not
work since it will be difficult to establish the correspondence
between more than two cross sections on neighboring slices.

%parallel: triangulation
To avoid the  disadvantages of the global approach,~\cite{BJ88,
MSS92, GB93, FL98, CD99} used the triangulation methods to compute
a volumetric mesh interpolating cross sections on each pair of
planes, such that correspondence between the curves can be
established and a surface mesh can be extracted. To avoid the
complicated calculation in fitting a volumetric
mesh,~\cite{BS96,BCL96,OPC96,KSS00,BGLS04,JWCET05,BV07} proposed to
project the cross sections from neighboring planes onto a common
plane and match the projections from different cross sections. The
final surface can then be built by connecting the curves and their
projections.

%parallel: implicit
Subsequent  works proposed various solutions to handling more
complicated cross sections on parallel planes. For examples, the
works~\cite{HZB92, CNKG02, TO99} extended the implicit surface
method which is often used in the point cloud fitting problem to fit
an implicit function to the input curves, extract the isosurface and
convert it to a mesh. Some works also considered the
normal~\cite{BMSdV10} or material~\cite{WD00} constraints during the
fitting process. However the global fitting method using implicit
function may produce werid local shapes due to the sparsity of the
cross section data. Tessellation of the isosurface should also be
taken care of to make sure the final surface still interpolate the
input curves, and the time for computation may be long.

%non-parallel
There have also  been some attempts made to reconstruct surfaces
from non-parallel cross sections.~\cite{RU90, WMTBDC95, BTS04}
extended the implicit function method to build a surface composed
of patches, each of which is bounded by cross sections on two
adjacent planes.~\cite{PT94,DP97,BM07} used a divide-and-conquer
scheme which first partitions the whole space into zones bounded by
the planes that the cross sections lie on. Then within each zone,
Delaunay triangulation is used to compute a sub-surface that
connects the cross sections on the faces of the zones. The
triangulation methods provide a feasible approach to build surfaces
from non-intersected cross sections, while for those who have
intersections or even more complex relationship, it will fail to
produce a satisfactory result.

A more recent  work~\cite{LBDLJ08}  adopted the divide-and-conquer
scheme and extended the method in~\cite{JWCET05} to handle more
complicated non-parallel cross sections. In each zone, the cross
sections are projected onto the medial axis planes to establish the
correspondence and sub-surface is built by connecting the cross
sections and their projections. The quality of the final surface is
then improved through iterative smoothing and refinement. This
method has been used in the reconstruction of 3D medical data from
contours of scanned 2D images~\cite{SLJGAGL09}. Although it can
handle non-parallel cross sections with more complex relationship,
the reconstruction result heavily relies on the geometric agency
(i.e., the medial axis plane). When the shape of the zone becomes
complex, the agency may become irregular, leading to unexpected
topological noise to the result surface.

%local method: multi-component problem
All these methods on handling the non-parallel cross sections aim at
building a single object. They used a standard partition scheme in
Computational Geometry to subdivide the 3D space into convex
arrangements~\cite{BM07}. However, this partition algorithm may
separate the connections of different surface components. As a
result, the final object will be composed of multiple disconnected
components, while in most cases a one-component surface is desired.
Contrarily,~\cite{EB11} proposed an approach to build multiple
objects from multi-labeled contours at the same time instead of one
single object, by defining rules to distinguish components belonging
to different objects. Both self-intersections and inter-object
intersections between these components can be avoided. While this is
useful in some bio-medical modeling applications such as
reconstructing neurons from electron microscopy images, the
multi-component problem when building one single object still
remains.

%post-edit after reconstruction
%Finally,  although all these algorithms focus on the reconstruction
%of 3D objects with satisfactory shapes, sometimes the topology of
%the result model still cannot meet the user's expectation. In that
%case, post-editing of the model is often needed, while these
%reconstruction algorithms can provide little useful information to
%this operation but only the 3D model itself. There have been some
%topology editing tools, especially those with simple sketch-based
%interfaces proposed~\cite{JZH07,JT09,HRABV11}. However, these
%methods all require to globally analyze and modify the model
%according to the user input, which seems complicated. In other
%words, the reconstruction and editing of the 3D model are regarded
%as independent operations. Our algorithm builds a bridge to connect
%them to help produce a 3D model with satisfactory shape in an
%efficient and natural manner.





































%\section{Digital geometry processing}
%\label{ch2:sec:dgp}
%
%With the development of 3D scanner techniques, digital geometry has become a new multimedia type after sound, image and video. It has wide applications in the area of engineering manufacturing, digital entertainment (game, movies, etc.), medicine and biology, art history and archeology, architecture design and so on. With the enlargement of its application area, there is a strong need for the processing techniques of digital geometry modeling, structure analysis and data optimization, leading to the emergence of digital geometry processing.
%
%Digital geometry describes 3D objects and uses 3D surfaces as the main representation. Surfaces in 3D space can be divided into the continuous and discrete forms. The continuous surfaces include parametric surfaces, implicit surfaces and subdivision surfaces. The discrete representations of surfaces include mesh and point cloud, and they are the most popular surface representations in digital geometry processing. This is due to their following advantages: they are relatively simple comparing to the continuous forms, both on the descriptions and transfers; they are the basic data types for the rendering of software and hardware; they are the output forms of most acquisition tools (CT, MRI, 3D scanner, etc.) and the inputs to most simulation/analysis tools.
%
%According to the pipeline of modeling, processing, application and transmission, the main research topics of digital geometry processing are stated as follows.
%
%First of all, since the acquisition from 3D scan equipment is point cloud which is quite irregular, the data needs to be processed for further applications. Point-based graphics provides methods of dealing with points directly, including the representation, modeling, processing and rendering of the points. Point-based graphics does not separate geometry and appearance/attributes of an object, and it seems natural for many 3D acquisition systems. Detailed introductions on point-based graphics can be found at~\cite{AGPPSZ04,GM09}.
%
%However since no connectivity or topology information is available in point cloud, mesh is often a more popular and desirable representation. To reconstruct a mesh from point cloud, two main methods are often used: the geometry-based method and implicit surface-based method. The former often uses the triangulation methods while the latter chooses to fit the points with implicit functions. The representative works of surface reconstruction from point cloud include~\cite{LC87,CL96,ABK98,HDDMS92,CBCMFME01}.
%
%Furthermore, since the point cloud data obtained from scanners may contain noise, difficulties will be brought to the surface reconstruction algorithms, leading to the geometric or topological errors of the reconstructed mesh. So mesh repairing is often needed. The geometric errors denote the holes, self-intersections or face overlaps of the mesh surface and the topological errors include the redundant handles, islands or cavities. The methods for mesh repairing include the surface-based methods~\cite{BS95,LP03,GW01} and volume-based methods~\cite{NT03,JT04,ZJH07}.
%
%Since the initially reconstructed mesh is often irregular, optimization techniques are needed to improve its quality. Mesh smoothing is an optimization method of fairing the mesh surface by only adjusting its geometry. It has multiple applications, such as mesh noise removal and multi-resolution analysis. It reduces curvature variations by moving the mesh vertices and is similar to the high frequency elimination in signal processing. Methods of mesh smoothing include the isotropic~\cite{TG95,DMSB99,LTJW07} and anisotropic filtering~\cite{CDR00,BX03,JDD03,SRML07}, and the selection of a proper algorithm should consider both the speed and the quality.
%
%Another method of regularizing the mesh is remeshing, which reconstructs the geometry and connectivity while striving to preserve the original geometry information. There have been methods proposed for remeshing the triangular mesh~\cite{AVDI03,SG03,AMD02} and quadrilateral mesh~\cite{ACDLD03,ML04}. By remeshing we could make a mesh meet our certain requirement (such as equilateral triangles) or represent the geometry better (such as minimize the number of triangles while preserving the sampling precision).
%
%Mesh simplification can reduce the number of vertices, faces and edges while keeping the original geometry as much as possible, so that the model could be rendered and processed on a computer with low performance. Algorithms of mesh simplification include the vertex-collapse~\cite{SZL92,CVMT96}, edge-collapse~\cite{HDDMS93,HH96,LT98}, and face-collapse~\cite{WHST01} methods. Currently the most popular method is the edge-collapse one which has largest degrees of freedom and best simplification effect.
%
%After a regular mesh is constructed, some further editing operations are often needed. One of the most important editing functions is mesh deformation, which deforms a mesh without changing the topology. A detailed review can be found in Chapter~\ref{ch2:sec:deformation}. Based on deformation, some advanced operations are developed, such as morphing, which generates a series of intermediate shapes between a mesh and its deformed shape and deformation transfer, which transfers the existing mesh deformation from one character to another.
%
%Since a mesh is often a two-manifold embedded in 3D space, comparing to images and videos, it does not have a regular parameter domain. This brings difficulties for the efficient processing of 3D meshes. Mesh parameterization studies the one-to-one mapping between a 3D surface and a simpler and more regular 2D parameter domain. Methods on parameterization include the base mesh parameterization, planar parameterization, spherical parameterization and so on. Parameterization has been used in almost each field of digital geometry modeling and processing. A detailed introduction on mesh parameterization can be found at~\cite{HLS07}.
%
%Given an existing model, there have been a lot of methods proposed for calculating and extracting its features~\cite{GE05,CRT04,PWYLH07,OBS04,RVLL08}. Based on these features, some more advanced operations such as mesh segmentation, shape matching and retrieval can be implemented.
%
%Finally after we have reconstructed and processed a mesh, transmission or storage of the mesh is needed. For complex meshes with lots of details, compression is necessary. This involves conversion between meshes and bit streams. Methods have been proposed for the encoding and decoding of meshes~\cite{JR99,SCT03,KSS00,AG05}, by applying the theory of information and coding.
%
%Digital geometry processing is a fast-growing area of research that uses concepts from applied mathematics, computer science and engineering to design efficient algorithms for the acquisition, reconstruction, analysis, manipulation, simulation and transmission of complex 3D models. For all the above mentioned research topics, a developing trend is towards the combination of various interactive techniques and algorithms for a more intelligent and effective processing. Among all the interactive techniques, sketch-based modeling has drawn much attention from scientists and researchers in recent years. We will then give a detailed review of sketch-based modeling in the following section.
